  %% Template pages
  \approvalpage
  \copyrightpage

  %% Acknowledgements
  \begin{acknowledgements}
    \addcontentsline{toc}{chapter}{Acknowledgements}

  First and foremost, I would like to express my immense gratitude toward my co-advisors, Professor Jarad Niemi and Professor Wendy Meiring. Jarad saw potential in me as a second-year graduate student and has provided me with invaluable professional and academic advice, even after he moved to Iowa to begin his new faculty position. Wendy, while playing an instrumental role in the department as graduate advisor, has always supported me with unwavering dedication. I am deeply grateful for the amount of time that they both have devoted to me, and I owe tremendous thanks to them for their continuous support over the past four years.

  I would also like to thank the remaining members serving on my thesis committee, Professor John Hsu and Professor Greg Ashby. John's warmth and kindness played a large part in my decision to come the UC-Santa Barbara, and Greg has been a tremendous help, sharing with me his expertise in the field of functional magnetic resonance imaging. In addition, I would like to thank Professor S. Rao Jammalamadaka, Professor Andrew Carter, Professor Yuedong Wang, and the entire faculty of the Department of Statistics and Applied Probability for their instruction and guidance throughout the development of my academic career.

  Finally, I would like to thank my loving family, friends, and fellow graduate students for all of their support and encouragement.

  \end{acknowledgements}
\ssp
  %% Vitae
  \begin{vitae}

    \addcontentsline{toc}{chapter}{Curriculum Vit\ae}
    {\small
      \begin{vitaesection}{Education}
        \vspace{-0.1cm}
      \item [2014] University of California, Santa Barbara, CA \\
      Doctor of Philosophy in Statistics and Applied Probability \\
      Ph.D. emphasis in Quantitative Methods in the Social Sciences
      \item [2010] University of California, Santa Barbara, CA \\
      Master of Arts in Mathematical Statistics
      \item [2009] University of Illinois at Urbana-Champaign, Champaign, IL \\
      Bachelor of Science in Statistics \\
      Secondary Major in History \\
      Minor in Computer Science
      \end{vitaesection}

      \begin{vitaesection}{Experience}
        \vspace{-0.1cm}
      \item [2013-2014] Summer Instructor, Department of Statistics and Applied Probability, University of California, Santa Barbara
      \item [2009-2014] Teaching Assistant, Department of Statistics and Applied Probability, University of California, Santa Barbara
      \item [2012] Statistics Consultant, Intellectual Ventures Laboratories, Epidemiological Modeling Group, Seattle
      \end{vitaesection}

      \begin{vitaesection}{Selected Publications}
        \vspace{-0.1cm}
      \item [2014] ``Comparison of the performance of particle filter algorithms applied to tracking of a disease epidemic'', with Jarad Niemi and Wendy Meiring. Mathematical Biosciences 255: 21-32.
      \item [2014] ``Large Loss Claims: The Market Shift Factor: Justification for a Statistical Solution'', with William Novotny in Journal of Advanced Appraisal Studies: 283-302.
        \vspace{0.3cm}
      \end{vitaesection}

      \begin{vitaesection}{Conference Presentations}
        \vspace{-0.1cm}
      \item [2014] ``Comparison of the performance of particle filter algorithms applied to tracking of a disease epidemic'', Joint Statistical Meetings
      \item [2013] ``Tracking and prediction of a disease epidemic using particle filtering'', WNAR Annual Meeting
        \vspace{0.3cm}
      \end{vitaesection}

      \begin{vitaesection}{Awards and Honors}
      \item [2010] Abraham Wald Memorial Award, UCSB Department of Statistics and Applied Probability
        \vspace{-0.1cm}
      \end{vitaesection}
    }
  \end{vitae}
\dsp

  %% Abstract

\begin{abstract}
    \addcontentsline{toc}{chapter}{Abstract}

We present contributions to epidemic tracking and analysis of fMRI data using sequential Monte Carlo methods within a state-space modeling framework. Using a model for tracking and prediction of a disease outbreak via a syndromic surveillance system, we compare the performance of several particle filtering algorithms in terms of their abilities to efficiently estimate disease states and unknown fixed parameters governing disease transmission. In this context, we demonstrate that basic particle filters may fail due to degeneracy when estimating fixed parameters, and we suggest the use of an algorithm developed by Liu and West (2001), which incorporates a kernel density approximation to the filtered distribution of the fixed parameters to allow for their regeneration. In addition, we show that seemingly uninformative uniform priors on fixed parameters can affect posterior inferences, and we suggest the use of priors bounded only by the support of the parameter. We demonstrate the negative impact of using multinomial resampling and suggest the use of either stratified or residual resampling within the particle filter. We also run a particle MCMC algorithm and show that the performance of the Liu and West (2001) particle filter is competitive with particle MCMC in this particular syndromic surveillance model setting. Finally, the improved performance of the Liu and West (2001) particle filter enables us to relax prior assumptions on model parameters, yet still provide reasonable estimates for model parameters and disease states.

We also analyze real and simulated fMRI data using a state-space formulation of a regression model with autocorrelated error structure. We demonstrate via simulation that analyzing autocorrelated fMRI data using a model with independent error structure can inflate the false positive rate of concluding significant neural activity, and we compare methods of accounting for autocorrelation in fMRI data by examining ROC curves. In addition, we show that comparing models with different autocorrelated error structures on the basis of the independence of fitted model residuals can produce misleading results. Using data collected from an fMRI experiment featuring an episodic word recognition task, we estimate parameters in dynamic regression models using maximum likelihood and identify clusters of low and high activation in specific brain regions. We compare alternative models for fMRI time series from these brain regions by approximating the marginal likelihood of the data using particle learning. Our results suggest that a regression model with a dynamic intercept is the preferred model for most fMRI time series in the episodic word recognition experiment within the brain regions we considered, while a model with a dynamic slope is preferred for a small percentage of voxels in these brain regions.

%    \abstractsignature
\end{abstract}





