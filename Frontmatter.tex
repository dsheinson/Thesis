  %% Template pages
  \approvalpage
  \copyrightpage

  %% Acknowledgements
  \begin{acknowledgements}
    \addcontentsline{toc}{chapter}{Acknowledgements}


  \end{acknowledgements}
\ssp
  %% Vitae
  \begin{vitae}

    \addcontentsline{toc}{chapter}{Curriculum Vit\ae}
    {\small
      \begin{vitaesection}{Education}
        \vspace{-0.1cm}
      \item [2014] Doctor of Philosophy in Statistics and Applied Probability, Department of
           Statistics and Applied Probability, University of California, Santa Barbara.
      \item [2009] Bachelor of Science in Statistics, University of Illinois at Urbana-Champaign, Champaign, IL.
      \end{vitaesection}

      \begin{vitaesection}{Experience}
        \vspace{-0.1cm}
      \item [2013-2014] Summer Instructor, Department of Statistics and Applied Probability, University of California, Santa Barbara
      \item [2009-2014] Teaching Assistant, Department of Statistics and Applied Probability, University of California, Santa Barbara
      \item [2012] Statistics Consultant, Intellectual Ventures Laboratories, Epidemiological Modeling Group, Seattle
      \end{vitaesection}

      \begin{vitaesection}{Selected Publications}
        \vspace{-0.1cm}
      \item [2014] ``Comparison of the performance of particle filter algorithms applied to tracking of a disease epidemic'', with Jarad Niemi and Wendy Meiring
      \item [2014] ``Large Loss Claims: The Market Shift Factor: Justification for a Statistical Solution'', with William Novotny
        \vspace{0.3cm}
      \end{vitaesection}

      \begin{vitaesection}{Conference Presentations}
        \vspace{-0.1cm}
      \item [2014] ``Comparison of the performance of particle filter algorithms applied to tracking of a disease epidemic'', Joint Statistical Meetings
      \item [2013] ``Tracking and prediction of a disease epidemic using particle filtering'', WNAR Annual Meeting
        \vspace{0.3cm}
      \end{vitaesection}

      \begin{vitaesection}{Awards and Honors}
      \item [2010] Abraham Wald Memorial Award
        \vspace{-0.1cm}
      \end{vitaesection}
    }
  \end{vitae}
\dsp

  %% Abstract

\begin{abstract}
    \addcontentsline{toc}{chapter}{Abstract}

We present contributions to epidemic tracking and analysis of fMRI data using sequential Monte Carlo methods within a state-space modeling framework. Using a model for tracking and prediction of a disease outbreak via a syndromic surveillance system, we compare the performance of several particle filtering algorithms in terms of their abilities to efficiently estimate disease states and unknown fixed parameters governing disease transmission. In this context, we show that basic particle filters may fail due to degeneracy in fixed parameter estimation and suggest the use of an algorithm developed by Liu and West (2001) that incorporates a kernel density approximation to the filtered distribution of the fixed parameters to allow for their regeneration. In addition, we show that seemingly uninformative uniform priors on fixed parameters can affect posterior inferences and suggest the use of priors bounded only by the support of the parameter. We show the negative impact of using multinomial resampling and suggest the use of either stratified or residual resampling within the particle filter. We also implement a particle MCMC algorithm and show that the performance of the Liu and West (2001) particle filter is competitive with particle MCMC in this particular model setting. Finally, we use this improved particle filtering methodology to relax prior assumptions on model parameters yet still provide reasonable estimates for model parameters and disease states.

We also analyze real and simulated fMRI data using a state-space formulation of a regression model with autocorrelated error structure. We show the negative impact of using a model with independent error structure on the false positive rate of concluding significant neural activity, and we compare methods of accounting for autocorrelation in fMRI data by examining the tradeoff between the false positive and true positive rates. In addition, we show that comparing models with different autocorrelated error structures on the basis of the independence of fitted model residuals can offer misleading results. Looking specifically at six brain regions of interest from an fMRI experiment featuring an episodic word recognition task, we estimate parameters in dynamic regression models using maximum likelihood estimation and identify clusters of low and high activation. Using a sequential Monte Carlo algorithm, we compare alternative models for fMRI time series by estimating the marginal likelihood of the data. Our results suggest that a regression model which incorporates both a dynamic slope and dynamic intercept is best suited for the data.

%    \abstractsignature
\end{abstract}





