\chapter{Future work \label{ch:future}}

In Chapters \ref{ch:epid}, \ref{ch:comp}, and \ref{ch:fmri}, we implemented a variety of SMC methods to estimate unobserved states and unknown fixed parameters in state-space models. Alternative methods exist that can perform more efficiently under certain model settings. For instance, Rao-Blackwellization \citep{Douc:Gods:Andr:on:2000} could have been implemented within the PL algorithm in order to marginalize states and track only state sufficient statistics. This would lead to more efficient estimates of the filtered distributions of unknown states. Future work on tracking epidemic outbreaks using SMC methods could incorporate Rao-Blackwellization to estimate unobserved disease states in a population more efficiently.

In Chapter \ref{ch:comp}, we found that the PL algorithm proved to be the more efficient than the RM or KDPF for analyzing data from the local level DLM described in Section \ref{sec:dlm:ll}. However, PL can only be applied to models for which the distributions $p(x_{t+1}|y_{t+1},x_t,\theta)$, $p(y_{t+1}|x_t,\theta)$, and $p(\theta|y_{1:t},x_{0:t})$ are analytically tractable. In many cases, only some of these distributions may be available, and it is also possible that only some elements of $\theta$ admit distributions that can be tracked using sufficient statistics. In this case, a strategy such as one described in \citet{dukic2012tracking} that combines the approaches described in Section \ref{sec:filtering} could be implemented to optimize efficiency by sampling states and fixed parameters from known distributions when possible and from approximations, as in \citet{Liu:West:comb:2001}, when not. This strategy could be used for the dynamic regression models described in Section \ref{sec:dlm:arwn} when stationarity of the state process is desired.

While SMC methods have an apparent advantage over MCMC by being able to sequentially update the estimate of the filtered distribution of the current state of a system, there are many reasons why an MCMC analysis might be preferred. For instance, the performance of SMC algorithms degrades if run over a long period of time, and SMC methods cannot operate on models where prior distributions on states or fixed parameters are too diffuse. To address these problems, MCMC and SMC methods could be used in conjunction with one another. For instance, MCMC could be run prior to running a particle filter in order to find a reasonable range of values over which prior distributions can be defined \cite[Chapter 5][]{petris:camp:2009:dynamic}. In addition, a possible strategy for continuously monitoring incoming streams of syndromic surveillance data could be to restart SMC runs daily using posterior samples from an MCMC run overnight.

SMC methods also have the ability to provide direct estimates of the marginal likelihood and thus compare alternative models. In chapter \ref{ch:fmri}, we used particle learning in this way to compare alternative models for fMRI data. However, our approach required separate particle filter runs under each model in order to obtain competing estimates of the marginal likelihood. In addition, a single PL run using 5000 particles on time series consisting of 245 TRs took about 45 minutes to complete. Thus, this model comparison strategy is only feasible for analyzing small brain regions of interest. There exist methods that can compare models within a single particle filter run by allowing particles to jump between models \citep{berz:gilks:rmcross:2001,zhou:joh:smcmodcomp:2013}. These approaches open up the possibility of comparing models of fMRI data from a larger portion of the brain within reasonable computing time.

It is likely that more complicated models for tracking an epidemic \citep{Sham:Kars:pnas:2012, Bhad:Ioni:mala:2011} and analyzing fMRI data \citep{buxton:balloon:1998} than what we presented in this thesis are needed to more accurately describe the data-generating mechanisms. For example, results from Chapter \ref{ch:fmri} indicate that a regression model with both a dynamic intercept and a dynamic slope may be appropriate for fMRI time series. The increase in the dimension of the parameter space associated with larger models makes estimation more challenging, as we saw in Sections \ref{sec:epid:extend} and \ref{sec:fmri:id}. While an MCMC approach such as PMCMC may perform better than SMC methods in some high-dimensional settings, both approaches are limited by the amount of data available. For this reason, spatio-temporal modeling approaches that borrow information from neighboring infected areas or brain regions are a promising direction for these fields.